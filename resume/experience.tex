\cvsection{Work Experience}
\begin{cventries}
  \cventry {Information Sciences Institute} {Research Programmer - Software Developer Consultant}
  {Los Angeles, USA} {Jan. 2019 - Oct 2020 / Apr 2021 - Present} { \begin{cvitems}\item{Developed APIs, pipelines, and workflows for integrating data from various sources using Python, FastAPI, Express, and PostgreSQL, cutting the model and data ingestion time from weeks to one day.} \item{Built a Model Catalog to enable interoperability between models, data, and existing resources using web semantic technologies, allowing MINT to find and run relevant data, models, or chains of models based on user-selected indicators (e.g., crop production).} \item{Created Helm charts for deploying services in Kubernetes clusters, including Information Science Institute, San Diego Supercomputer Center, and Texas Advanced Computing Center, reducing installation time to 10 minutes and providing clear documentation.} \item{Designed, integrated, and maintained CI/CD pipelines using GitHub Actions, enabling developers to efficiently test and deploy changes.} \item{Deployed and managed authentication services using Keycloak, enabling login from different agencies via OIDC.} \item{Implemented the provenance extension to DISK, improving data tracking and documentation, which helps neuroscientists better understand experiment results.}\end{cvitems} }
  \cventry {The University of Texas at Austin} {Software Developer Consultant} {Austin, TX}
  {Mar 2021 - Present} { \begin{cvitems}\item{Propose, design, and implement software solutions for Planet Texas 2050, an eight-year initiative aimed at enhancing community resilience. Collaborate with experts from diverse fields including architecture, city planning, public health, geology, and engineering, to develop integrated and innovative solutions.} \item{Refactored metadata APIs to enhance performance and maintainability using LoopBack 4, TypeScript, OpenAPI, and PostgreSQL, providing clear API specifications to agencies and detecting inconsistencies between the metadata schema and the API.} \item{Integrated a new execution system for running simulations from MINT, allowing to use 5,800 nodes with approximately 175,000 CPU cores and optional NVIDIA A100 GPUs.} \item{Designed and built a web application using FastAPI and React, enabling users to run simulations on High Performance Computing (HPC) clusters without needing SSH access.}\end{cvitems} }

  \cventry {Universidad Técnica Federico Santa María} {Part-time Teacher}
  {Valparaíso, Chile} {Jul 2016 - Dec 2018}
  { \begin{cvitems}\item{Proposed and teach a course on Software Deployment on Linux, covering tools such as Docker, AWS, GitHub Actions and Nginx. This course equips students with practical skills for deploying applications.} \item{Taught an Operating Systems course, guiding students through core concepts with practical exercises in a Linux environment and related DevOps tools.}\end{cvitems} }

  % \cventry
  % {Chilean Virtual Observatory (ChiVO)}
  % {DevOps Engineer}
  % {Valparaíso, Chile}
  % {Jan 2018 - May 2018}
  % {
  %   \begin{cvitems}
  %     \item{Worked on the FONDEF IT15I10041 project to implement a virtual observatory in Chile, adhering to standards and protocols defined by the International Virtual Observatory Alliance (IVOA).}
  %     \item{Proposed and implemented a cloud service for astronomers to use Jupyter notebooks in a high-performance processing environment using Docker Containers and Kubernetes.}
  %   \end{cvitems}
  % }

  \cventry {Linets} {Systems Engineer} {Santiago, Chile} {Jan. 2014 - Mar. 2016}
  { \begin{cvitems}\item{Managed system administration for 400 servers, achieving improved performance and reliability through effective troubleshooting, configuration, database administration, and optimization.} \item{Led the deployment of the first OpenStack environment in Chile, supporting Beebop, the country's first public cloud.} \item{Automated the OpenStack deployment process using Docker Containers and Ansible playbooks, streamlining operations and reducing deployment time from hours to minutes.} \item{Implemented Ceph-based shared storage for the public cloud, enhancing data performance, reliability, and scalability, which improved storage efficiency by 40\%.}\end{cvitems} }

  % \cventry
  %   {Computer Science Department}
  %   {Systems Administrator}
  %   {Valparaíso, Chile}
  %   {Aug. 2011 - Oct. 2014}
  %   {
  %     \begin{cvitems}
  %       \item{Administered internal servers, ensuring the availability of services for students and faculty.}
  %     \end{cvitems}
  %   }
\end{cventries}