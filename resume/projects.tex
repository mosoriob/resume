\cvsection{Projects}
\begin{cventries}
  \cventry {Developer} {\href{https://mintproject.readthedocs.io/en/stable/}{MINT Project}}
  {Los Angeles, USA} {Jan 2019 - Present} { \begin{cvitems}\item {MINT assists an analyst to easily use sophisticated simulation models and data in order to explore the role of weather and climate in water on food availability in select regions of the world.} \item {MINT has been applied by The Defense Advanced Research Projects Agency (DARPA) to the assessment of food insecurity in Ethiopia, where 1.2 million simulations were run to allow experts to simulate food production.} \item {MINT has been adopted by the Planet Texas 2050 project, which aims to design solutions that will make our Texas communities stronger, more resilient, and better prepared for current and future challenges} \item {MINT has been used by Supercomputer Center at the University of California, San Dieg to simulate the impact of wildfires in California.}\end{cvitems} }

  \cventry {Co-creator}
  {\href{https://github.com/KnowledgeCaptureAndDiscovery/OBA}{OBA}} {}
  {May 2019 - Present}
  { \begin{cvitems}\item {OBA reads ontologies (OWL) and generates an OpenAPI Specification (OAS). Using this definition, OBA creates a REST API server automatically.} \item {Nominated for the Best Paper Resources Award at the 2020 International Semantic Web Conference (ISWC).}\end{cvitems} }
  % \cventry
  %   {Creator}
  %   {DockerPedia}
  %   {Valparaíso, Chile}
  %   {Feb. 2018 - March 2020}
  %   {
  %     \begin{cvitems}
  %       \item {DockerPedia automatically describes the software components of Docker images and their vulnerabilities using Clair.}
  %       \item {Dockerpedia provides a visualization tool, which uses the available data, to help Docker users search and compare between different Docker images, allowing them to find software distributions which fit their needs and monitor the state of Docker repositories over time. The project has been deprecated.}
  %     \end{cvitems}
  %   }
\end{cventries}