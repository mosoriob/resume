\cvsection{Experiencia Laboral}
\begin{cventries}

  \cventry
    {Information Sciences Institute}
    {Research Programmer - Software Developer Consultant}
    {Los Ángeles, EE. UU.}
    {Ene. 2019 - Oct. 2020 / Abr. 2021 - Presente}
    {
      \begin{cvitems}
        \item{Desarrollador Principal de MINT Project. Proyecto financiado por Defense Advanced Research Projects Agency en marco de la iniciativa World Modelers (13 millones USD), que busca mejorar la capacidad de los analistas para comprender y predecir eventos complejos en el mundo real.}
        \item{Desarrollé APIs, pipelines y interfaces de usuarios para integrar datos de diversas base de datos utilizando Python, Node y PostgreSQL, reduciendo el tiempo de creación de modelos y datos de semanas a un día.}
        \item{Desarrollé sistemas computacionales para facilitar la interoperabilidad entre modelos y datos, proporcionando información clave para la toma de decisiones sobre distintos experimentos, por ejemplo, la producción de alimentos en Etiopía y su impacto en la población en función de las condiciones climáticas, sin requerir conocimientos computacionales por parte de los analistas.}
        \item{Desarrollé y mantengo instaladores (Helm Charts) para desplegar servicios en clústeres de Kubernetes, incluyendo el Centro de Supercomputación de San Diego y el Centro de Computación Avanzada de Texas, reduciendo el tiempo de instalación desde días a 10 minutos y proporcionando documentación clara.}
        \item{Diseñé, integré y mantuve pipelines de CI/CD utilizando GitHub Actions, permitiendo a los desarrolladores probar y desplegar cambios de manera eficiente.}
      \end{cvitems}
    }

  \cventry
    {The University of Texas at Austin}
    {Software Developer Consultant}
    {Austin, TX}
    {Mar. 2021 - Presente}
    {
      \begin{cvitems}
        \item{Propongo, diseño e implemento soluciones de software para Planet Texas 2050, una iniciativa destinada a mejorar la resiliencia de Texas, USA. Colaboro con expertos de diversos campos incluyendo arquitectura, planificación urbana, salud pública, geología e ingeniería, para desarrollar soluciones centradas en el usuario e innovadoras.}
        \item{Refactoricé APIs de metadatos para mejorar el rendimiento y la mantenibilidad, proporcionando especificaciones claras de API a las agencias y detectando inconsistencias entre el esquema de metadatos y la API.}
        \item{Integré un nuevo sistema de ejecución para ejecutar simulaciones desde MINT, permitiendo utilizar 5,800 nodos con aproximadamente 175,000 núcleos de CPU y GPUs NVIDIA A100.}
        \item{Diseñé y construí una aplicación web utilizando FastAPI y React, permitiendo a los usuarios ejecutar simulaciones en clústeres de Computación de Alto Rendimiento (HPC) de manera remota sin necesidad de acceso SSH.}
      \end{cvitems}
    }

  \cventry
    {Linets}
    {Ingeniero de Sistemas}
    {Santiago, Chile}
    {Ene. 2014 - Mar. 2016}
    {
      \begin{cvitems}
        \item{Gestioné la administración de sistemas para 400 servidores, logrando mejorar el rendimiento y la fiabilidad mediante la resolución efectiva de problemas, configuración, administración de bases de datos y optimización.}
        \item{Lideré el despliegue del primer entorno OpenStack en Chile, dando soporte a Beebop, la primera nube pública del país.}
        \item{Automaticé el proceso de despliegue de OpenStack utilizando Contenedores Docker y playbooks de Ansible, agilizando las operaciones y reduciendo el tiempo de despliegue de horas a minutos.}
        \item{Implementé almacenamiento compartido basado en Ceph para la nube pública, mejorando el rendimiento, la fiabilidad y la escalabilidad de los datos, lo que mejoró la eficiencia de almacenamiento en un 40\%.}
      \end{cvitems}
    }

\end{cventries}