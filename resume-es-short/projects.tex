\cvsection{Proyectos}
\begin{cventries}
  \cventry
    {Desarrollador}
    {\href{https://mintproject.readthedocs.io/en/stable/}{Proyecto MINT}}
    {Los Ángeles, EE. UU.}
    {Ene. 2019 - Presente}
    {
      \begin{cvitems}
        \item {MINT asiste a un analista para utilizar fácilmente modelos de simulación sofisticados y datos con el fin de explorar el papel del clima en el agua y la disponibilidad de alimentos en regiones seleccionadas del mundo.}
        \item {MINT ha sido aplicado por Defense Advanced Research Projects Agency (DARPA) para la evaluación de la inseguridad alimentaria en Etiopía, donde se realizaron 1.2 millones de simulaciones para permitir a los expertos simular la producción de alimentos.}
        \item {MINT ha sido adoptado por el proyecto Planet Texas 2050, que tiene como objetivo diseñar soluciones que harán que nuestras comunidades de Texas sean más fuertes, más resilientes y estén mejor preparadas para los desafíos actuales y futuros.}
        \item {MINT ha sido utilizado por el Centro de Supercomputación de la Universidad de California, San Diego para simular el impacto de los incendios forestales en California.}
      \end{cvitems}
    }
  \cventry
      {Co-creador}
      {\href{https://github.com/KnowledgeCaptureAndDiscovery/OBA}{OBA}}
      {}
      {Mayo 2019 - Presente}
      {
        \begin{cvitems}
          \item {OBA lee ontologías (OWL) y genera una Especificación OpenAPI (OAS). Utilizando esta definición, OBA crea automáticamente un servidor de API REST.}
          \item {Nominado al Premio al Mejor Artículo de Recursos en la Conferencia Internacional de Web Semántica (ISWC) 2020.}
        \end{cvitems}
      }
\end{cventries}