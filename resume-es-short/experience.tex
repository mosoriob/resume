\cvsection{Experiencia Laboral}
\begin{cventries}

  \cventry
    {Information Sciences Institute}
    {Research Programmer - Software Developer Consultant}
    {Los Ángeles, EE. UU.}
    {Ene. 2019 - Oct. 2020 / Abr. 2021 - Presente}
    {
      \begin{cvitems}
        \item{Desarrollé APIs, pipelines y flujos de trabajo para integrar datos de diversas fuentes utilizando Python, FastAPI, Express y PostgreSQL, reduciendo el tiempo de ingesta de modelos y datos de semanas a un día.}
        \item{Construí un catalogo para permitir la interoperabilidad entre modelos, datos y recursos existentes utilizando tecnologías web semánticas, permitiendo a MINT encontrar y ejecutar datos, modelos o cadenas de modelos relevantes basados en indicadores seleccionados por el usuario (por ejemplo, producción de cultivos).}
        \item{Creé Helm charts para desplegar servicios en clústeres de Kubernetes, incluyendo el Instituto de Ciencias de la Información, el Centro de Supercomputación de San Diego y el Centro de Computación Avanzada de Texas, reduciendo el tiempo de instalación a 10 minutos y proporcionando documentación clara.}
        \item{Diseñé, integré y mantuve pipelines de CI/CD utilizando GitHub Actions, permitiendo a los desarrolladores probar y desplegar cambios de manera eficiente.}
        \item{Desplegué y gestioné servicios de autenticación utilizando Keycloak, permitiendo el inicio de sesión desde diferentes agencias vía OIDC.}
      \end{cvitems}
    }

  \cventry
    {The University of Texas at Austin}
    {Software Developer Consultant}
    {Austin, TX}
    {Mar. 2021 - Presente}
    {
      \begin{cvitems}
        \item{Propongo, diseño e implemento soluciones de software para Planet Texas 2050, una iniciativa de ocho años destinada a mejorar la resiliencia comunitaria. Colaboro con expertos de diversos campos incluyendo arquitectura, planificación urbana, salud pública, geología e ingeniería, para desarrollar soluciones integradas e innovadoras.}
        \item{Refactoricé APIs de metadatos para mejorar el rendimiento y la mantenibilidad utilizando LoopBack 4, TypeScript, OpenAPI y PostgreSQL, proporcionando especificaciones claras de API a las agencias y detectando inconsistencias entre el esquema de metadatos y la API.}
        \item{Integré un nuevo sistema de ejecución para ejecutar simulaciones desde MINT, permitiendo utilizar 5,800 nodos con aproximadamente 175,000 núcleos de CPU y GPUs NVIDIA A100 opcionales.}
        \item{Diseñé y construí una aplicación web utilizando FastAPI y React, permitiendo a los usuarios ejecutar simulaciones en clústeres de Computación de Alto Rendimiento (HPC) sin necesidad de acceso SSH.}
      \end{cvitems}
    }

  \cventry
    {Linets}
    {Ingeniero de Sistemas}
    {Santiago, Chile}
    {Ene. 2014 - Mar. 2016}
    {
      \begin{cvitems}
        \item{Gestioné la administración de sistemas para 400 servidores, logrando mejorar el rendimiento y la fiabilidad mediante la resolución efectiva de problemas, configuración, administración de bases de datos y optimización.}
        \item{Lideré el despliegue del primer entorno OpenStack en Chile, dando soporte a Beebop, la primera nube pública del país.}
      \end{cvitems}
    }

\end{cventries}